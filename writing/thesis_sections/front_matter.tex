\let\origparident\parindent
\setlength{\parindent}{0pt}

{\LARGE
\vspace*{2cm}
\textbf{\thetitle}
}

\cleardoublepage

\begin{minipage}[c][\textheight]{.7\textwidth}
{\itshape
I hereby declare that the thesis submitted is my own, unaided work, completed without any
unpermitted external help. Only the sources and resources listed were used.
\vspace{1em}
}

\begin{otherlanguage}{ngerman}
Hiermit erkläre ich, dass ich die vorliegende Arbeit selbstständig und eigenhändig
sowie ohne unerlaubte fremde Hilfe und ausschließlich unter Verwendung der aufgeführten
Quellen und Hilfsmittel angefertigt habe.
\vspace{2em}
\end{otherlanguage}

Berlin,

\vspace{3em}
\dotfill{}
\end{minipage}


\cleardoublepage

\vspace*{2cm}

{\Huge
\textbf{\thetitle}
}

\vspace{2cm}

{\large
\textbf{\theauthor}
}

\vspace*{\fill}

A thesis submitted to the\\
\textbf{Faculty of Electrical Engineering and Computer Science}\\
of the\\
\textbf{Technical University of Berlin}\\
in partial fulfillment of the requirements for the degree\\
\textbf{Bachelor of Science} in \textbf{Computer Science}\\[1em]

Berlin, Germany\\
\thedate\\[1em]
\includelogo{}

\newpage

\vspace*{\fill}

Main supervisor:

Prof. Dr. habil. Odej Kao, Technical University of Berlin\\[1em]

Research advisor:

Dr. Alexander Acker, Technical University of Berlin

\begin{otherlanguage}{ngerman}
\chapter*{\vspace{-2cm}Zusammenfassung}
\thispagestyle{empty}

\vspace{-0.5cm}

% Kurze Zusammenfassung der Arbeit in 150-250 Wörtern.
Das Verstehen und Eindämmung der Auswirkungen von Fehlern und Ausfällen ist ein unverzichtbarer Aspekt für den Betrieb moderner \Ac{it}-Infrastrukturen,
jedoch macht es alleine der Umfang großer Systeme schwierig, Probleme festzustellen und nachzuvollziehen.
Sogenannte Logdateien beinhalten beträchtliche Mengen an Informationen bezüglich des Zustandes eines Systems,
allerdings erfordert eine manuelle Untersuchung erhebliche Zeitressourcen und ein fundiertes Verständnis vieler Komponenten;
mit steigender Komplexität des Systems wird dies nicht leichter.
Um diese Analyse zu vereinfachen, wird in dieser Arbeit die Verwendung vortrainierter Modelle zur Verarbeitung menschlicher Sprache vorgeschlagen,
um Logdaten zusammenzufassen.
Die Annahme ist, dass eine kompaktere Beschreibung der relevanten Ererignisse eine manuelle Untersuchung einer Logdatei beschleunigen kann,
und dass solche Systeme einen besseren Überblick bieten können, wenn Benutzer mit Logdaten interagieren.
Weiterhin werden zwei Verfahren zur halb-automatischen Generierung von Zusammenfassungen vorgestellt,
um Datensätze zu entwickeln, auf denen Vorgehen zur Zusammenfassung von Logdaten ausgewertet werden können.
Der vorgeschlagene Ansatz ist in der Lage, ein bisher existierendes System zur Zusammenfassung von Logdaten
auf mehreren zuvor untersuchten Datensätzen zu übertreffen.
Insgesamt ist eine durchschnittliche Steigerung der \acs*{rouge}-1 \(F_1\) Metrik um \(12\) Punkte zu beobachten.
Dies zeigt, dass moderne Sprachverarbeitungsmodelle zumindest teilweise befähigt sind Logdaten zu verstehen,
sodass sie in Zukunft Anwendung bei der Analyse von Logdateien finden können.
\end{otherlanguage}

\vspace{-1.5cm}

{\let\cleardoublepage\relax \chapter*{Abstract}}
\thispagestyle{empty}

\vspace{-0.5cm}

% Short version of the thesis in 150-250 words.
Understanding and mitigating the impact of failures is essential to operating modern \ac{it} infrastructures,
though their large scale alone makes it challenging to diagnose any problems.
Logs supply crucial information on system state and behavior,
but analyzing them manually requires substantial time investments and in-depth knowledge about the system;
it becomes difficult for human operators to understand them with rising system complexity.
To ease the analysis, we suggest using pre-trained \ac*{nlp} models to summarize logs
and propose two summarization tasks to construct reference summaries from logs semi-automatically.
We assume that a concise representation of the relevant events in a log speeds up the analysis conducted by human operators.
Additionally, such summarization systems may provide a better overview anywhere users interact with log-data.
Our approach performs comparative to previous work on previously studied datasets for log summarization
but outperforms it significantly on datasets the previous approach finds more difficult:
Overall we achieve improved \acs*{rouge}-1 \(F_1\) scores of \(12\) points on average.
This shows that current \acs*{nlp} models are at least in part able to transfer their knowledge to the domain of log-data;
Thus, their application may prove helpful to future research regarding log analysis.

\cleardoublepage

\tableofcontents

\newpage

\listoffigures
\listoftables
\printacronyms[heading=chapter*, name=List of Acronyms]

\setlength{\parindent}{\origparident}
